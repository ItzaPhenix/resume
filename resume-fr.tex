\documentclass[11pt,a4paper,sans]{moderncv}
\usepackage[utf8]{inputenc}
\usepackage[english]{babel}
\usepackage{multicol}
\moderncvstyle{classic}
\moderncvcolor{orange}
\nopagenumbers{}
%\usepackage[scale=0.80]{geometry}
\usepackage[margin=0.8in]{geometry}
\usepackage{moderntimeline}
\tlmaxdates{2011}{2016}

\firstname{Guillaume}
\familyname{Dupuis}

\title{Etudiant ingénieur IT - Stage de fin d'étude}
\address{13 Allée du gros Chêne}{Verneuil - 78480 - France}
\email{g.dupuis.tourniaire@gmail.com}
%\homepage{phenixnest.com}
\mobile{+33~(0)~6~38~53~64~09}
\social[linkedin]{gdupuistourniaire}
%\social[github]{itzaphenix}


\quote{Ingénieur passionné d'informatique, principalement intéressé sur le
développement système, les systèmes embarqués, l'infrastructure et l'IoT}

\begin{document}
\makecvtitle

\vspace*{-10mm}

\section{Formation}
\tlcventry{2011}{2016}{Ingénieur IT}{EPITA}{Ecole d'ingénieur
informatique}{Etudiant}{
\begin{itemize}
    \item \textbf{Spécialisation}: (GISTRE) Génie Informatique des Systèmes
        Temps Réel et Embarqués
\end{itemize}
}
\tldatecventry{2013}{Echange}{Stafford University}{Stafford}{Computer science
master}{}

\section{Expérience}
\tldatecventry{2013}{Electricien \& service
support}{AFP}{Paris}{Stage}{Electrique service support pour les bureaux et les
datacenters de l'AFP (Agence France Presse)}
\tldatecventry{2014}{Etudiant en R\&D}{Thales Communication \& Security}{Paris}
{Stage}{Portage d'Android dans un conteneur LXC sécurisé sur architecture ARM
\begin{itemize}
\item \textbf{Technologies utilisées:} LXC, Android, Debian, debootstrap
\end{itemize}
}
\tlcventry{2014}{2016}{Assistant enseignant en informatique}{EPITA}{}{}{
\begin{itemize}
\item \textbf{Compétences enseignées:} Shell, C, C++, Java, SQL et modèles de données pour
    300 étudiants
\end{itemize}
}

\subsection{Projets d'école}
\tldatecventry{2013}{Malloc}{C}{}{Implémentation de malloc de la libc}{}
\tldatecventry{2013}{42sh}{C}{}{Shell posix en C fait en équipe de
5. J'ai fait les modules ligne de commande, le prompt, l'exécution, les
built-ins et
\texttt{fnmatch()} tous écrits depuis zéro.
}{}
\tldatecventry{2014}{Tiger}{C++}{}{Développement d'un compilateur de langage
    tiger en équipe de 3}{
\begin{itemize}
\item \textbf{Technologies utilisées:} Flex, Bison
\end{itemize}
}
\tldatecventry{2015}{STOS}{C/ASM}{}{noyau x86 pour apprendre les interruptions
et la gestion mémoire}{}
\tldatecventry{2015}{Domobox}{Python}{}{Box d'IoT modulaire pour monitorer des
modules domestiques}{
\begin{itemize}
\item \textbf{Technologies utilisées:} Python3, Raspberry pi 2, arduino, MongoDB
\end{itemize}
}

\subsection{Associations}
\tlcventry{2013}{0}{Bureau des élèves}{}{}{}{Membre du Bureau des élèves de l'EPITA}
\tlcventry{2013}{2015}{Staff Com}{}{}{}{Membre de l'équipe de communication de
l'EPITA}

\section{Compétences informatiques}
\cvitem{\textbf{OS:}}{\textbf{Linux}, Unix, Windows}
\cvitem{\textbf{Langages:}}{\textbf{C}, C++, C\#, Java, Python, Shell, ASM x86, ADA}
\cvitem{\textbf{Outils:}}{\textbf{Git}, gdb, valgrind, \textbf{vim}, Visual
Studio, MS Project, MicroEj}

\section{Langues}
\cvitem{\textbf{Francais:}}{Langue maternelle}
\cvitem{\textbf{Anglais:}}{Compétences professionnelles}

\end{document}
