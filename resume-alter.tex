\documentclass[11pt,a4paper,sans]{moderncv}
\usepackage[utf8]{inputenc}
\usepackage[english]{babel}
\usepackage{multicol}
\moderncvstyle{classic}
\moderncvcolor{orange}
\nopagenumbers{}
%\usepackage[scale=0.80]{geometry}
\usepackage[margin=0.8in, top=1.5cm]{geometry}
\usepackage{moderntimeline}
\tlmaxdates{2011}{2016}

\firstname{Guillaume}
\familyname{Dupuis}

\title{CS student at EPITA - Looking for a 6 months engineering internship}
\address{13 Allee du gros Chene}{Verneuil - 78480 - France}
\email{g.dupuis.tourniaire@gmail.com}
%\homepage{phenixnest.com}
\mobile{+33~(0)~6~38~53~64~09}
\social[linkedin]{gdupuistourniaire}
%\social[github]{itzaphenix}

\quote{Passionnate computer scientist, mostly interested in system
developpement, embedded system, infrastructure and IoT}
\begin{document}
\makecvtitle

\vspace*{-10mm}

\section{Formation}
\tlcventry{2011}{2016}{CS Engineer}{EPITA}{French engineering school}{Student}{
\begin{itemize}
    \item Specialization: (GISTRE) Computing Engineering for Real-time and
        Embedded Systems
\end{itemize}
}
\tldatecventry{2013}{Exchange - 6 months}{Stafford University}{Stafford}{Computer science
master}{}

\section{Experience}
\tlcventry{2015}{2016}{Teacher assistant in computer science}{EPITA}{1-Year}{}{
    Member of a team \textbf{30+ teaching assistants} and taught \textbf{300+
    students}. The experience of assistant is broken down into two periods :
    YAKA (January-August) and ACU (September-January)
\begin{itemize}
\item Teaching: Shell, C/C++, UNIX, Java, SQL technologies and design patterns
\end{itemize}
}
\tldatecventry{2014}{R\&D student}{Thales Communication \& Security}{Paris}
{Internship}{Developed secure embedded systems in light virtualization for
    mobile device. Porting \textbf{Android} in a \textbf{LXC container} with
    Debian on Galaxy Note 2 and Nexus 7
\begin{itemize}
    \item Improvement: Provided \textbf{2+ new devices} to show the capability
    to port Android to LXC on ARM devices
\item Concepts learned: ARM architecture, embedded system on mobile,
    network on linux, Android system architecture (aosp), boot procedure of
    linux
\item Technologies used: LXC, Android, Linux Debian, Android, multistrap
\end{itemize}
}
\tldatecventry{2013}{Electrician \& service support}{AFP}{Paris}{Internship}{Gave
electrical service support for desks and datacenter of AFP (French Press
Agency)
\begin{itemize}
\item Improvement: Reduced the risk of blackout of 50\% for servers
    with new power supply's organisation
\item Concepts learned: Organization of international structures,
    power supplies of high avaibility server, work in a structure team
\end{itemize}
}

\subsection{School projects}
\tldatecventry{2015}{STOS}{C/ASM}{}{x86 modular monolithic kernel intended to
    teach interruptions (events), memory management (pagination) and memory
protection, managed by LSE (System \& Security Lab)}{}
\tldatecventry{2014}{Tiger}{C++}{}{Developped a compiler of tiger language in
teams of 3, managed by LRDE (research lab)}{
\begin{itemize}
    \item \textbf{Technologies used:} Flex, Bison, Python, design patterns
\end{itemize}
}
\tldatecventry{2013}{42sh}{C}{}{A posix shell in C done in course in teams of
    5. I mostly did the command line and prompt (with termcap), execution,
    built-in and \texttt{fnmatch()} all written from scratch}{}
\tldatecventry{2013}{Malloc}{C}{}{Implementation of malloc, free, calloc and
    realloc of libc with optimized first fit and binary buddies algorithms}{}

\section{Computer skills}
\cvitem{\textbf{OS:}}{\textbf{Linux}, Unix, Windows}
\cvitem{\textbf{Languages:}}C\#, {\textbf{C}, C++, \textbf{Java}, Shell,
    \textbf{Python}, ASM x86, ADA, SQL}
\cvitem{\textbf{Tools:}}{\textbf{Git}, gdb, valgrind, Visual Studio, MS Project}

\end{document}
